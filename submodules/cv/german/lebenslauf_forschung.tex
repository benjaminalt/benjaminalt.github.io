%%%%%%%%%%%%%%%%%
% This is an example CV created using altacv.cls (v1.1.3, 30 April 2017) written by
% LianTze Lim (liantze@gmail.com), based on the
% Cv created by BusinessInsider at http://www.businessinsider.my/a-sample-resume-for-marissa-mayer-2016-7/?r=US&IR=T
%
%% It may be distributed and/or modified under the
%% conditions of the LaTeX Project Public License, either version 1.3
%% of this license or (at your option) any later version.
%% The latest version of this license is in
%%    http://www.latex-project.org/lppl.txt
%% and version 1.3 or later is part of all distributions of LaTeX
%% version 2003/12/01 or later.
%%%%%%%%%%%%%%%%

%% If you want to use \orcid or the
%% academicons icons, add "academicons"
%% to the \documentclass options.
%% Then compile with XeLaTeX or LuaLaTeX.
% \documentclass[10pt,a4paper,academicons]{altacv}

%% Use the "normalphoto" option if you want a normal photo instead of cropped to a circle
\documentclass[10pt,a4paper,normalphoto]{../template/altacv}

%% AltaCV uses the fontawesome and academicon fonts
%% and packages.
%% See texdoc.net/pkg/fontawecome and http://texdoc.net/pkg/academicons for full list of symbols.
%% When using the "academicons" option,
%% Compile with LuaLaTeX for best results. If you
%% want to use XeLaTeX, you may need to install
%% Academicons.ttf in your operating system's font %% folder.


% Change the page layout if you need to
\geometry{left=1cm,right=1cm,marginparwidth=-1.2cm,marginparsep=1.2cm,top=1cm,bottom=1cm}

% Change the font if you want to.

% If using pdflatex:
\usepackage[utf8]{inputenc}
\usepackage[T1]{fontenc}
\usepackage[default]{lato}
\usepackage[ngerman]{babel}

% If using xelatex or lualatex:
% \setmainfont{Lato}

% Change the colours if you want to
\definecolor{Blue}{HTML}{707070}
\definecolor{SlateGrey}{HTML}{2E2E2E}
\definecolor{LightGrey}{HTML}{666666}
\colorlet{heading}{Blue}
\colorlet{accent}{Blue}
\colorlet{emphasis}{SlateGrey}
\colorlet{body}{LightGrey}

% Change the bullets for itemize and rating marker
% for \cvskill if you want to
\renewcommand{\itemmarker}{{\small\textbullet}}
\renewcommand{\ratingmarker}{\faCircle}

%% sample.bib contains your publications
\addbibresource{publications.bib}

\begin{document}
\name{Benjamin Alt}
% \tagline{Informatikstudent \& Softwareentwickler}
\photo{3.2cm}{../bewerbungsfoto_cropped}
\personalinfo{%
  % Not all ofs these are required!
  % You can add your own with \printinfo{symbol}{detail}
  \mailaddress{Welfenstraße 1, 76137 Karlsruhe}
  \location{Karlsruhe, Deutschland}
  \phone{+49 160 8548454}
  \email{benjamin\_alt@outlook.com}
  \homepage{benjaminalt.github.io}
  \github{github.com/benjaminalt}
  \linkedin{linkedin.com/in/benjamin-alt}
}

%% Make the header extend all the way to the right, if you want.
\begin{fullwidth}
\makecvheader
\end{fullwidth}

\vspace{-15pt}

%% Provide the file name containing the sidebar contents as an optional parameter to \cvsection.
%% You can always just use \marginpar{...} if you do
%% not need to align the top of the contents to any
%% \cvsection title in the "main" bar.
\cvsection{Bildungsweg}

\cvsection{Education}

\cvsuperevent{University of Bremen}{Bremen, Germany}
\cvsubevent{Ph.D. Computer Science}{cum laude}{2020 - 2025}
\begin{itemize}
  \item Dissertation: \textit{Neurosymbolic Robot Programming - A Framework for AI-Enabled Programming of Robot Manipulation Tasks} (\href{https://doi.org/10.26092/elib/3727}{\faLink ~PDF})
  % \item Committee: Prof. Michael Beetz, Prof. Aude Billard, Prof. Tanja Schultz, Prof. Nico Hochgeschwendner
  \item Advisor: Prof. Michael Beetz, AICOR Institute for Artificial Intelligence
\end{itemize}

\medskip

\cvsuperevent{Karlsruhe Institute of Technology}{Karlsruhe, Germany}
\cvsubevent{M.Sc. Computer Science}{with distinction}{2017 -- 2019}
\begin{itemize}
  \item Thesis: \textit{Automatic Parameterization of Robot Programs via Learning of Neural Program Representations}
  \item Areas of Specialization: Robotics and Automation; Anthropomatics and Cognitive Systems
  \item Merit scholarship of the German Acad. Scholarship Foundation (Studienstiftung des deutschen Volkes)
\end{itemize}

\medskip

\cvsubevent{B.Sc. Computer Science}{}{2015 -- 2017}
\begin{itemize}
  \item Thesis: \textit{Machine Learning for Pose Optimization: An Integrated Framework for the Development and Monitoring of Adaptive Robot Programs}
\end{itemize}

\medskip

\cvsuperevent{Institut d'Études Politiques de Paris (SciencesPo)}{Reims, France}
\cvsubevent{B.A. Political Science}{summa cum laude}{2012 -- 2015}{Reims, France}
\begin{itemize}
  \item Areas of Specialization: Law, History, Economics
  \item 2014-2015 at Princeton University with a focus on Mathematics \& Computer Science
\end{itemize}

%\clearpage

% \iffalse
\cvsection{Publikationen}

\nocite{*}

% \printbibliography[heading=pubtype,title={\printinfo{\faBook}{Books}},type=book]

% \divider

% \printbibliography[heading=pubtype,title={\printinfo{\faFileTextO}{Journal Articles}}, type=article]

% \divider

\printbibliography[heading=pubtype,title={\printinfo{\faGroup}{Conference Proceedings}},type=inproceedings]

%% If the NEXT page doesn't start with a \cvsection but you'd
%% still like to add a sidebar, then use this command on THIS
%% page to add it. The optional argument lets you pull up the
%% sidebar a bit so that it looks aligned with the top of the
%% main column.
% \addnextpagesidebar[-1ex]{page3sidebar}

% \fi

\cvsection{Arbeitserfahrung}

\cvevent{Research Scientist}{ArtiMinds Robotics}{Jan 2023 - heute}{Karlsruhe, Deutschland}
\begin{itemize}
  \item TODO
\end{itemize}

\cvevent{Research Scientist}{ArtiMinds Robotics}{Okt 2019 - Dez 2022}{Karlsruhe, Deutschland}
\begin{itemize}
  \item Grundlagenforschung zu Sequence-to-Sequence-Translation über kontinuierlichen Räumen \& Inversion neuronaler Netze
  \item Angewandte Forschung und prototypische Vorentwicklung für die datengetriebene Optimierung von Fertigungsprozessen
  \item Beantragung und Durchführung von Förderprojekten (BMBF, ZIM) sowie Kommunikation mit Projektpartnern
  \item Betreuung von Abschlussarbeiten
\end{itemize}

\cvevent{Junior Software Engineer}{ArtiMinds Robotics}{Sep 2017 -- Aug 2019}{Karlsruhe, Deutschland}
\begin{itemize}
  \item Entwicklung einer Lösung zur datengetriebenen, automatischen Optimierung von Robotertrajektorien
  \item Initiale Konzeption und Mitarbeit an einer Industrie-4.0-Plattform für die Darstellung \& Analyse von Prozessdaten
  \item \textit{Associate Trainer}: Schulungen für Kunden und Distributoren
\end{itemize}

\cvevent{Werkstudent}{ArtiMinds Robotics}{Jan 2017 -- Aug 2017}{Karlsruhe, Deutschland}
\begin{itemize}  
  \item Bachelorarbeit: Dynamische Posenoptimierung eines industriellen Manipulators mit von rekurrenten neuronalen Netzen
  \item Entwicklung eines Algorithmus zur Berechnung von Sicherheitsebenen aus Robotertrajektorien
\end{itemize}

\divider

\cvevent{Wissenschaftliche Hilfskraft}{Institut für Prozessautomation und Robotik, KIT}{Sep 2016 -- Dez 2016}{Karlsruhe, Deutschland}
\begin{itemize}  
  \item Entwicklung einer ROS-basierten Kalibrierungslösung für einen Kraft-Momenten-Sensor
  \item Entwicklung eines Java-Codegenerators für die AutomationML-Ontologie
\end{itemize}

\divider

\cvevent{Werkstudent}{WIBU Systems AG}{Jun 2015 -- Aug 2016}{Karlsruhe, Deutschland}
\begin{itemize}  
  \item Entwicklung und Einrichtung einer virtualisierten Testumgebung mit VmWare ESXi und Jenkins CI
  \item Entwicklung interner Webanwendungen und SW-Prototypen mit HTML5/PHP/Javascript
  \item Mitentwicklung eines grafischen API-Guides/Codegenerators in C++/Qt5
\end{itemize}

\cvsection{Open-Source-Projekte}

\cvopensourceproject{data-fusion}{github.com/KIT-ISAS/data-fusion}
\begin{itemize}
  \item Implementierung \& Simulation von Algorithmen zur Datenfusion in Sensornetzwerken
\end{itemize}

\cvopensourceproject{Human Brain Project: Neurorobotics Platform}{}
\begin{itemize}
  \item Steuerung eines sechsbeinigen Laufroboters mit Spiking Neural Networks\hfill \githublink{github.com/benjaminalt/hbpprak\_locomotion}
  \item Object Tracking mit Spiking Neural Networks\hfill \githublink{github.com/Scaatis/hbpprak\_perception}
\end{itemize}

\cvopensourceproject{ST Verification Studio}{github.com/VerifAPS/stvs}
\begin{itemize}
  \item Java-basierte, grafisch unterstützte formale Verifikation industrieller SPS-Programme in Structured Text
\end{itemize}


\cvsection{Auszeichnungen}

\cvaward{Stipendium der Studienstiftung des deutschen Volkes}{Mär 2018 -- Sep 2019}
\cvaward{e-fellows Stipendium der Deutschen Telekom, Georg v. Holtzbrinck \& McKinsey}{Jun 2012 -- Sep 2019}
\cvaward{\textit{Summa Cum Laude Avec Felicitations du Jury} für das akad. Jahr 13/14, IEP de Paris (SciencesPo)}{Jun 2014}
\cvaward{\textit{Summa Cum Laude} für das akad. Jahr 12/13, IEP de Paris (SciencesPo)}{Jun 2013}
\cvaward{Scheffelpreis (Deutsch) \& Auszeichnung der Fachschaft für Mathematik (Fichte-Gymnasium)}{Jun 2012}
\cvaward{Glemser Award for Future Excellency}{Jun 2012}


\cvsection{Sprachkenntnisse}

\cvskill{Deutsch, Englisch}{10}{}
\cvskill{Französisch}{8}{}
\cvskill{Spanisch}{4}{}
\cvskill{Latein, Italienisch}{2}{}

\cvsection{Kulturelles Engagement}
\cvevent{Übersetzung (Englisch-Deutsch) und freie Mitarbeit}{Literaturverlag GolubBooks}{Okt 2011 -- heute}{Karlsruhe, Deutschland}

\medskip
Karlsruhe, den {\today} \hspace{30mm} % $\vcenter{\hbox{\includegraphics[height=13mm]{signature}}}$
\end{document}