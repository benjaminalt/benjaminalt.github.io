\cvevent{Research Scientist}{ArtiMinds Robotics}{Jan 2023 - heute}{Karlsruhe, Deutschland}
\begin{itemize}
  \item TODO
\end{itemize}

\cvevent{Research Scientist}{ArtiMinds Robotics}{Okt 2019 - Dez 2022}{Karlsruhe, Deutschland}
\begin{itemize}
  \item Grundlagenforschung zu Sequence-to-Sequence-Translation über kontinuierlichen Räumen \& Inversion neuronaler Netze
  \item Angewandte Forschung und prototypische Vorentwicklung für die datengetriebene Optimierung von Fertigungsprozessen
  \item Beantragung und Durchführung von Förderprojekten (BMBF, ZIM) sowie Kommunikation mit Projektpartnern
  \item Betreuung von Abschlussarbeiten
\end{itemize}

\cvevent{Junior Software Engineer}{ArtiMinds Robotics}{Sep 2017 -- Aug 2019}{Karlsruhe, Deutschland}
\begin{itemize}
  \item Entwicklung einer Lösung zur datengetriebenen, automatischen Optimierung von Robotertrajektorien
  \item Initiale Konzeption und Mitarbeit an einer Industrie-4.0-Plattform für die Darstellung \& Analyse von Prozessdaten
  \item \textit{Associate Trainer}: Schulungen für Kunden und Distributoren
\end{itemize}

\cvevent{Werkstudent}{ArtiMinds Robotics}{Jan 2017 -- Aug 2017}{Karlsruhe, Deutschland}
\begin{itemize}  
  \item Bachelorarbeit: Dynamische Posenoptimierung eines industriellen Manipulators mit von rekurrenten neuronalen Netzen
  \item Entwicklung eines Algorithmus zur Berechnung von Sicherheitsebenen aus Robotertrajektorien
\end{itemize}

\divider

\cvevent{Wissenschaftliche Hilfskraft}{Institut für Prozessautomation und Robotik, KIT}{Sep 2016 -- Dez 2016}{Karlsruhe, Deutschland}
\begin{itemize}  
  \item Entwicklung einer ROS-basierten Kalibrierungslösung für einen Kraft-Momenten-Sensor
  \item Entwicklung eines Java-Codegenerators für die AutomationML-Ontologie
\end{itemize}

\divider

\cvevent{Werkstudent}{WIBU Systems AG}{Jun 2015 -- Aug 2016}{Karlsruhe, Deutschland}
\begin{itemize}  
  \item Entwicklung und Einrichtung einer virtualisierten Testumgebung mit VmWare ESXi und Jenkins CI
  \item Entwicklung interner Webanwendungen und SW-Prototypen mit HTML5/PHP/Javascript
  \item Mitentwicklung eines grafischen API-Guides/Codegenerators in C++/Qt5
\end{itemize}