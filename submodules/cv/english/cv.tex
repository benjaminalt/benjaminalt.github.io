%%%%%%%%%%%%%%%%%
% This is an example CV created using altacv.cls (v1.1.3, 30 April 2017) written by
% LianTze Lim (liantze@gmail.com), based on the
% Cv created by BusinessInsider at http://www.businessinsider.my/a-sample-resume-for-marissa-mayer-2016-7/?r=US&IR=T
%
%% It may be distributed and/or modified under the
%% conditions of the LaTeX Project Public License, either version 1.3
%% of this license or (at your option) any later version.
%% The latest version of this license is in
%%    http://www.latex-project.org/lppl.txt
%% and version 1.3 or later is part of all distributions of LaTeX
%% version 2003/12/01 or later.
%%%%%%%%%%%%%%%%

%% If you want to use \orcid or the
%% academicons icons, add "academicons"
%% to the \documentclass options.
%% Then compile with XeLaTeX or LuaLaTeX.
% \documentclass[10pt,a4paper,academicons]{altacv}

%% Use the "normalphoto" option if you want a normal photo instead of cropped to a circle
\documentclass[10pt,a4paper,normalphoto]{../template/altacv}

%% AltaCV uses the fontawesome and academicon fonts
%% and packages.
%% See texdoc.net/pkg/fontawecome and http://texdoc.net/pkg/academicons for full list of symbols.
%% When using the "academicons" option,
%% Compile with LuaLaTeX for best results. If you
%% want to use XeLaTeX, you may need to install
%% Academicons.ttf in your operating system's font %% folder.


% Change the page layout if you need to
\geometry{left=1cm,right=1cm,marginparwidth=-1.2cm,marginparsep=1.2cm,top=1cm,bottom=1cm}

% Change the font if you want to.

% If using pdflatex:
\usepackage[utf8]{inputenc}
\usepackage[T1]{fontenc}
\usepackage[default]{lato}
\usepackage[english]{babel}
\usepackage{hyperref}
\usepackage[sfdefault]{roboto}
\usepackage{orcidlink}

% If using xelatex or lualatex:
% \setmainfont{Lato}

% Change the colours if you want to
\definecolor{Blue}{HTML}{707070}
\definecolor{SlateGrey}{HTML}{2E2E2E}
\definecolor{LightGrey}{HTML}{666666}
\colorlet{heading}{Blue}
\colorlet{accent}{Blue}
\colorlet{emphasis}{SlateGrey}
\colorlet{body}{black}

% Change the bullets for itemize and rating marker
% for \cvskill if you want to
\renewcommand{\itemmarker}{{\small\textbullet}}
\renewcommand{\ratingmarker}{\faCircle}

%% sample.bib contains your publications
\addbibresource{../publications.bib}

\AtEveryBibitem{% Clean up the bibtex rather than editing it
 \clearlist{address}
 \clearfield{date}
%  \clearfield{eprint}
 \clearfield{isbn}
 \clearfield{issn}
 \clearlist{location}
 \clearfield{month}
 \clearfield{series}
 \clearfield{url}
 \clearfield{urldate}
 \clearfield{urlyear}
 \clearfield{urlmonth}
 \clearfield{urlday}

 
%  \ifentrytype{book}{}{% Remove publisher and editor except for books
%   \clearlist{publisher}
%   \clearname{editor}
%  }
}

\begin{document}

\name{Benjamin Alt}
\tagline{Technical Director, AICOR Institute for Artificial Intelligence}
% \photo{3.2cm}{bewerbungsfoto_cropped}
\personalinfo{%
  % Not all ofs these are required!
  % You can add your own with \printinfo{symbol}{detail}
  % \mailaddress{Welfenstraße 1, 76137 Karlsruhe}
  % \location{Karlsruhe, Germany}
  \phone{+49 421 21864016}
  \email{benjamin.alt@uni-bremen.de}
  \orcid{0009-0002-8790-1671}
  \homepage{\href{https://benjaminalt.github.io/}{benjaminalt.github.io}}
  \github{\href{https://github.com/benjaminalt}{benjaminalt}}
  \linkedin{\href{https://linkedin.com/in/benjamin-alt}{linkedin.com/in/benjamin-alt}}
}

%% Make the header extend all the way to the right, if you want.
\begin{fullwidth}
\makecvheader
\end{fullwidth}

\vspace{-5pt}

%% Provide the f ile name containing the sidebar contents as an optional parameter to \cvsection.
%% You can always just use \marginpar{...} if you do
%% not need to align the top of the contents to any
%% \cvsection title in the "main" bar.

\cvsection{Relevant Experience}

\cvsuperevent{AICOR Institute for Artificial Intelligence, University of Bremen}{Bremen, Germany}
\cvsubevent{Technical Director}{}{Apr 2025 - today}
\begin{itemize}
  \item Develop an ecosystem for open-source cognitive robotics
  \item Establish a deep and broad network of research and industry partners
  \item Acquire funding for research and technology transfer
  \item Coordinate technology transfer activities
\end{itemize}

\medskip

\cvsuperevent{ArtiMinds Robotics}{Karlsruhe, Germany}
\cvsubevent{Senior Team Lead Research}{}{Oct 2024 - Mar 2025}
\begin{itemize}
  \item Led a team of 7 full-time and student researchers
  \item Coordinated AI technology transfer in customer projects and commercial product development
  \item Established and expanding long-term research partnerships with >20 academic institutions and >15 industry partners
  \item Led 8 publicly funded research projects on cognitive robotics with >2M € of grant volume
  \item Acquired >800k € of grant volume for 2 publicly funded research projects on advanced industrial robotics
\end{itemize}

\medskip

\cvsubevent{Senior Research Scientist}{}{Jan 2023 - Sep 2024}
\begin{itemize}
  \item Researched and published on scalable, interpretable artificial intelligence for industrial robots (8 conference papers)
  \item Acquired and realized 5 publicly funded research projects in excess of 1.4M € of grant volume
  \item Conducted in-house consulting on AI methods, applications and technology transfer
  \item Mentored and supervised 14 graduate and undergraduate students
\end{itemize}

\medskip

\cvsubevent{Research Scientist}{}{Oct 2019 - Dec 2022}
\begin{itemize}
  \item Researched and published on semi-symbolic robot program inference with deep neural networks (5 conference papers, 2 book chapters)
  \item Implemented and patented a commercial AI solution for the data-driven optimization of industrial production processes
  \item Acquired and realized 6 publicly funded research projects in excess of 1.5M € of grant volume
  \item Mentored and supervised 16 graduate and undergraduate students
\end{itemize}

% \medskip

% \cvsubevent{Junior Software Engineer}{}{Sep 2017 - Aug 2019}
% \begin{itemize}
%   \item Developed a solution for data-driven robot program optimization
%   \item Bootstrapped and co-developed a commercial platform for the aggregation, display and analysis of robot process data
%   \item \textit{Associate Trainer}: Training and education of industry customers
% \end{itemize}

% \cvevent{Student Engineer}{ArtiMinds Robotics}{Jan 2017 -- Sep 2017}{Karlsruhe, Germany}
% \begin{itemize}  
%   \item Bachelor's Thesis': Online pose optimization of an industrial robotic manipulator with recurrent neural networks
%   \item Development of an algorithm for the automatic generation of safety regions from robot trajectories
% \end{itemize}

% \divider

% \cvevent{Research Assistant}{Institute for Process Automation and Robotics, KIT}{Sep 2016 -- Dec 2016}{Karlsruhe, Germany}
% \begin{itemize}  
%   \item Development of a ROS-based solution for shape-from-motion force-torque sensor calibration
%   \item Development of a Java code generator for the AutomationML ontology
% \end{itemize}

% \divider

% \cvevent{Student Engineer}{WIBU Systems AG}{Jun 2015 -- Aug 2016}{Karlsruhe, Germany}
% \begin{itemize}  
%   \item Development and deployment of a virtualized software testbed with VmWare ESXi and Jenkins CI
%   \item Development of internal web applications with HTML5/PHP/JavaScript
%   \item Development of a graphical API guide and code generator in C++/Qt5
% \end{itemize}

\cvsection{Education}

\cvsuperevent{University of Bremen}{Bremen, Germany}
\cvsubevent{Ph.D. Computer Science}{cum laude}{2020 - 2025}
\begin{itemize}
  \item Dissertation: \textit{Neurosymbolic Robot Programming - A Framework for AI-Enabled Programming of Robot Manipulation Tasks} (\href{https://doi.org/10.26092/elib/3727}{\faLink ~PDF})
  % \item Committee: Prof. Michael Beetz, Prof. Aude Billard, Prof. Tanja Schultz, Prof. Nico Hochgeschwendner
  \item Advisor: Prof. Michael Beetz, AICOR Institute for Artificial Intelligence
\end{itemize}

\medskip

\cvsuperevent{Karlsruhe Institute of Technology}{Karlsruhe, Germany}
\cvsubevent{M.Sc. Computer Science}{with distinction}{2017 -- 2019}
\begin{itemize}
  \item Thesis: \textit{Automatic Parameterization of Robot Programs via Learning of Neural Program Representations}
  \item Areas of Specialization: Robotics and Automation; Anthropomatics and Cognitive Systems
  \item Merit scholarship of the German Acad. Scholarship Foundation (Studienstiftung des deutschen Volkes)
\end{itemize}

\medskip

\cvsubevent{B.Sc. Computer Science}{}{2015 -- 2017}
\begin{itemize}
  \item Thesis: \textit{Machine Learning for Pose Optimization: An Integrated Framework for the Development and Monitoring of Adaptive Robot Programs}
\end{itemize}

\medskip

\cvsuperevent{Institut d'Études Politiques de Paris (SciencesPo)}{Reims, France}
\cvsubevent{B.A. Political Science}{summa cum laude}{2012 -- 2015}{Reims, France}
\begin{itemize}
  \item Areas of Specialization: Law, History, Economics
  \item 2014-2015 at Princeton University with a focus on Mathematics \& Computer Science
\end{itemize}

\cvsection{Selected Publications}

\nocite{*}

\printbibliography[heading=pubtype,title={Conference Papers},type=inproceedings,keyword=relevant]  

% \divider

\printbibliography[heading=pubtype,title={Journal Articles},type=article,keyword=relevant]

% \divider

\printbibliography[heading=pubtype,title={Book Chapters}, type=incollection,keyword=relevant]

% \divider

\printbibliography[heading=pubtype,title={Patents},type=patent,keyword=relevant]

\printbibliography[heading=pubtype,title={Preprints},type=misc,keyword=relevant]

Full list of publications: {\href{https://nc.uni-bremen.de/index.php/s/soQcLeMCaY8XYgT}{\faLink ~benjaminalt.github.io/publications}

\cvsection{Projects and Grants}
\project{RESI-TSN}{Resilient, Intelligent Time-Sensitive Networks}{BMBF}{2024-2026}{Co-PI}
\project{AutoKOOP}{Automated Workflow for the Patient-Specific Carbon-Reinforced Orthopedic Products}{BMBF}{2024-2026}{Co-PI}
\project{ConfidentAM}{Continuous Fiber-Reinforced Low-Density Thermoplastic Additive Manufacturing}{BMBF}{2023-2026}{Co-PI}
\project{VADER}{Networked Ditigal Assistant for the Data-Driven Engineering of Robot Workcells}{BMWK}{2023-2025}{Co-PI, Collaborator}
\project{Next2OEM}{A Digital and Automated Value Chain for Wiring Harness Manufacturing}{BMWK}{2023-2025}{Co-PI, Collaborator}
\project{EASY}{Efficient Analysis and Control in a Dynamic Edge-Cloud-Continuum for Industrial Production}{BMWK}{2023-2025}{Co-PI, Collaborator}
\project{GANResilRob}{Generative Adversarial Networks and Semantics for Resilient, Flexible Robotic Production}{BMWK}{2022-2025}{Co-PI, Collaborator}
\project{KARL}{Competency Center Artificial Intelligence for Working and Living in the Karlsruhe Region}{BMBF}{2021-2025}{Co-PI, Collaborator}
% \project{Koala-Grasp}{Cognitively Assisted Laparoscopy: A Learning Robotic Assistance System for Surgical Grasping and Holding Tasks}{BMBF}{2021-2024}{Collaborator}
% \project{RoboGrind}{Hybrid AI for Flexible Robotic Surface Treatment}{InvestBW}{2021-2023}{Collaborator}
% \project{HoLLiECares}{A Multifunctional Service Robot to Support Care Professionals in the Hospital}{BMBF}{2020-2023}{Collaborator}
% \project{KIRK}{AI-based Robot Calibration}{BMBF}{2020-2022}{Collaborator}
% \project{ILIAS}{Imitation Learning from Human Demonstrations in Virtual Reality for Physical Human-Robot Interaction in Assistance Tasks}{BMBF}{2019-2022}{Collaborator}
% \project{MonRob}{Frameworks and Technologies for the Monitoring, Analysis and Online Adaption of Industrial Robotic Production Processes}{ZIM}{2017-2018}{Collaborator}
% \project{RoPHa}{Robust Perception Skills for Robotic Household Assistance in the Context of Elderly Care}{BMBF}{2017-2020}{Collaborator}
% \project{ProBot}{Proactive Diagnosis and Conception of Collaborative Robot
% Deployment in Small and Medium-Sized
% Enterprises}{BMBF}{2019-2022}{Collaborator}
\newline
6 additional BMBF projects, 1 ZIM project and 1 InvestBW project as collaborator.

%% If the NEXT page doesn't start with a \cvsection but you'd
%% still like to add a sidebar, thñen use this command on THIS
%% page to add it. The optional argument lets you pull up the
%% sidebar a bit so that it looks aligned with the top of the
%% main column.
% \addnextpagesidebar[-1ex]{page3sidebar}

% \cvsection{Awards and Scholarships}

\cvaward{Scholarship of the German Acad. Scholarship Foundation (Studienstiftung des deutschen Volkes)}{Mar 2018 -- Sep 2019}
\cvaward{e-fellows Scholarship (Deutsche Telekom, Georg v. Holtzbrinck \& McKinsey)}{Jun 2012 -- Sep 2019}
% \cvaward{Scheffel award (German) \& faculty award for mathematics (Fichte Gymnasium)}{Jun 2012}
% \cvaward{Glemser Award for Future Excellency}{Jun 2012}

% \cvsection{Open Source Projects}

% \cvopensourceproject{data-fusion}{github.com/KIT-ISAS/data-fusion}
% \begin{itemize}
%   \item Implementation and simulation of algorithms for data fusion in distributed sensor networks
% \end{itemize}

% \cvopensourceproject{Human Brain Project: Neurorobotics Platform}{}
% \begin{itemize}
%   \item Control of a six-legged walking robot with spiking neural networks \hfill \githublink{github.com/benjaminalt/hbpprak\_locomotion}
%   \item Object tracking with spiking neural networks \hfill \githublink{github.com/Scaatis/hbpprak\_perception}
% \end{itemize}

% \cvopensourceproject{ST Verification Studio}{github.com/VerifAPS/stvs}
% \begin{itemize}
%   \item Java-based graphical formal verification of PLC programs in Structured Text
% \end{itemize}

% \divider

% \cvsection{Skills}

\cvlist{Robotics}{Task and motion planning, force control, 3D visual perception, robot programming, human-robot interaction, model predictive control, manipulation of deformable objects}

\medskip

\cvlist{Machine learning}{Deep learning, imitation learning, learning from demonstration, differentiable programming, model-based optimization, interpretability, informed machine learning}

\medskip

\cvlist{Research management}{Grant acquisition, science communication, stakeholder management, technology transfer, strategic planning}

\medskip

\cvlist{Leadership}{Team leadership, mentoring, talent acquisition}

\divider

\cvlist{Programming languages}{Python (8 years of professional experience), C++ (3 years), Prolog (1 year), Java}

\medskip

\cvlist{Development tools}{Git, DVC, Jira, CMake, Jenkins CI}

\medskip

\cvlist{Frameworks}{PyTorch, NumPy, Keras, ROS, Qt}

% \divider

% \cvlist{German, English}{C2}
% \cvlist{French}{B2}
% \cvlist{Spanish}{A2}
% \cvlist{Latin, Italian}{A1}

% \cvsection{Cultural Engagement}
% \cvevent{Translation (English-German) and freelance work}{Golub Books Publishing}{Oct 2011 -- Sep 2021}{Karlsruhe, Germany}

% \cvsection{Volunteer Work}
% \cvevent{Selection Committee Member}{Studienstiftung des deutschen Volkes}{Nov 2022 -- today}{Karlsruhe, Germany}

\bigskip
\bigskip

Bremen, {\today}\par%\hspace{30mm} $\vcenter{\hbox{\includegraphics[height=13mm]{signature}}}$
\end{document}