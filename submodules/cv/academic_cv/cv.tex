%%%%%%%%%%%%%%%%%%%%%%%%%%%%%%%%%%%%%%%%%
% Compact Academic CV
% LaTeX Template
% Version 2.0 (6/7/2019)
%
% This template originates from:
% https://www.LaTeXTemplates.com
%
% Authors:
% Dario Taraborelli (http://nitens.org/taraborelli/home)
% Vel (vel@LaTeXTemplates.com)
%
% License:
% CC BY-NC-SA 3.0 (http://creativecommons.org/licenses/by-nc-sa/3.0/)
%
%%%%%%%%%%%%%%%%%%%%%%%%%%%%%%%%%%%%%%%%%

%----------------------------------------------------------------------------------------
%	PACKAGES AND OTHER DOCUMENT CONFIGURATIONS
%----------------------------------------------------------------------------------------

\documentclass[11pt]{article} % Default document font size
\usepackage{comment}
\usepackage{currfile}

\usepackage[style=ieee,maxnames=20,backend=biber]{biblatex}
\AtEveryBibitem{\clearfield{urldate}}
\DeclareFieldFormat{labelnumber}{}
\setlength{\biblabelsep}{0pt}
\defbibenvironment{bibliography}
  {\list{}{\setlength{\leftmargin}{0pt}}}
  {\endlist}
  {\item}

\addbibresource{../publications.bib}
\DeclareFieldFormat{urldate}{}
\renewbibmacro*{url+urldate}{%
  \printfield{url}}

\input{\currfiledir structure.tex} % Include the file specifying the document structure and styling

% Set PDF meta-information
\hypersetup{
	pdftitle={Benjamin Alt - Curriculum vitae},
	pdfauthor={Benjamin Alt}
}

%----------------------------------------------------------------------------------------

\begin{document}

%----------------------------------------------------------------------------------------
%	CONTACT AND GENERAL INFORMATION
%----------------------------------------------------------------------------------------

{\LARGE\bfseries Benjamin Alt}

\bigskip

Curriculum Vitae

\bigskip\medskip % Whitespace

\begin{minipage}{.5\textwidth}
Albert-Nestler-Str. 11\\
76131 Karlsruhe, Germany\\
+49 721 509998-66\\
\href{mailto:benjamin.alt@artiminds.com}{benjamin.alt@artiminds.com}
\end{minipage}%
% \begin{minipage}{.5\textwidth}
% \begin{flushright}
% Welfenstr. 1\\
% 76137 Karlsruhe, Germany\\
% +49 1608548454\\
% \href{mailto:benjamin.alt@uni-bremen.de}{benjamin.alt@uni-bremen.de}
% \end{flushright}
% \end{minipage}

\medskip % Whitespace

Personal website: \href{https://benjaminalt.github.io}{benjaminalt.github.io} % Academic/personal website

\medskip % Whitespace

Born: September 16, 1993---Karlsruhe, Germany\\ % Date of birth
Nationality: German % Nationality

%------------------------------------------------

\section*{Current Position}

\emph{Senior Team Lead Research}, ArtiMinds Robotics, Karlsruhe, Germany % Current or most recent employment position

%------------------------------------------------

\section*{Areas of Specialisation}

Artificial Intelligence, Cognitive Robotics, AI Safety, Human-AI Interaction % Primary areas of research interest


%----------------------------------------------------------------------------------------
%	EDUCATION
%----------------------------------------------------------------------------------------

\section*{Education}

\years{2020-2025}\textsc{Ph.D.} Computer Science\\
Institute for Artificial Intelligence, University of Bremen, Germany\\
Dissertation: \textit{Neurosymbolic Robot Programming: A Framework for AI-Enabled Programming of Robot Manipulation Tasks}

\medskip

\years{2017-2019}\textsc{M.Sc.} Computer Science\\
Institute for Anthropomatics and Robotics, Karlsruhe Institute of Technology, Germany

\medskip

\years{2015-2017}\textsc{B.Sc.} Computer Science\\
Institute for Anthropomatics and Robotics, Karlsruhe Institute of Technology, Germany

\medskip

\years{2012-2015}\textsc{B.A.} Political Science\\
Institut d'Études Politiques de Paris, France and Princeton University, NJ, USA

%----------------------------------------------------------------------------------------
%	WORK EXPERIENCE
%----------------------------------------------------------------------------------------

\section*{Professional Experience}

\years{2024-}Senior Team Lead Research, ArtiMinds Robotics, Karlsruhe, Germany

\medskip

\years{2023-2024}Senior Research Scientist, ArtiMinds Robotics, Karlsruhe, Germany

\medskip

\years{2019-2022}Research Scientist, ArtiMinds Robotics, Karlsruhe, Germany

\medskip

\years{2017-2019}Junior Software Engineer, ArtiMinds Robotics, Karlsruhe, Germany

\medskip

\years{2016-2017}Student Engineer, ArtiMinds Robotics, Karlsruhe, Germany

\medskip

\years{2015-2016}Student Engineer, WIBU Systems AG, Karlsruhe, Germany

%----------------------------------------------------------------------------------------
%	PUBLICATIONS AND TALKS
%----------------------------------------------------------------------------------------

\nocite{*}
\setlength{\bibitemsep}{0pt} % Remove spacing between bibliography items
\setlength{\bibparsep}{0pt} % Remove spacing within bibliography blocks
\setlength{\bibhang}{0pt}   % No hanging indentation for bibliography items

\newcommand{\firstpubentries}[2]{%
  \defbibcheck{#1}{\iffieldequalstr{year}{#1}{}{\skipentry}}%
  \years{#1}\vspace{0pt}\printbibliography[type=#2,heading=none,check=#1]
}

\newcommand{\pubentries}[2]{%
  \defbibcheck{#1}{\iffieldequalstr{year}{#1}{}{\skipentry}}%
  \years{#1}\vspace{-10pt}\printbibliography[type=#2,heading=none,check=#1]
}


\section*{Publications}

% Books, Edited Volumes, Refereed Journal Articles, Book Chapters, Conference Proceedings, Encyclopedia Entries, Book Reviews, Manuscripts in Submission (give journal title), Manuscripts in Preparation

\subsection*{Book Chapters}

\firstpubentries{2022}{incollection}%
\pubentries{2020}{incollection}

\subsection*{Refereed Journal Articles}
\firstpubentries{2024}{article}

\subsection*{Conference Proceedings}

\firstpubentries{2025}{inproceedings}
\pubentries{2024}{inproceedings}
\pubentries{2023}{inproceedings}
\pubentries{2022}{inproceedings}
\pubentries{2021}{inproceedings}
\pubentries{2020}{inproceedings}

% \subsection*{Manuscripts in Submission}
% \firstpubentries{2024}{misc}

%----------------------------------------------------------------------------------------
%	GRANTS, HONOURS AND AWARDS
%----------------------------------------------------------------------------------------

\section*{Awards \& Honors}

\years{2018-2019}Merit scholarship of the German Acad. Scholarship Foundation (Studienstiftung des deutschen Volkes)\\
\years{2012-2019}e-fellows Merit Scholarship (Deutsche Telekom, Georg v. Holtzbrinck \& McKinsey)

\section*{Grants}

\years{2024-2026}German Ministry of Education and Research grant \#16KIS2236\\
Resilient, Intelligent Time-Sensitive Networks (RESI-TSN)\\
Co-PI\\
\years{2024-2026}German Ministry of Education and Research grant \#13GW0735C\\
Automated Workflow for the Patient-Specific Carbon-Reinforced Orthopedic Products (AutoKOOP)\\
Co-PI\\
\years{2023-2026}German Ministry of Education and Research grant \#03LB3097A\\
Continuous Fiber-Reinforced Low-Density Thermoplastic Additive Manuf. (ConfidentAM)\\
Co-PI (from 10/2024)\\
\years{2023-2025}German Ministry of Economic Affairs and Climate Action grant \#13IK017A\\
Networked Ditigal Assistant for the Data-Driven Engineering of Robot Workcells (VADER)\\
PI (from 10/2024), Collaborator\\
\years{2023-2025}German Ministry of Economic Affairs and Climate Action grant \#13IK026A\\
A Digital and Automated Value Chain for Wiring Harness Manufacturing (Next2OEM)\\
Co-PI (from 10/2024), Collaborator\\
\years{2023-2025}German Ministry of Economic Affairs and Climate Action grant \#01MD22002B\\
Efficient Analysis and Control in a Dyn. Edge-Cloud-Continuum for Industr. Production (EASY)\\
Co-PI (from 10/2024), Collaborator\\
\years{2022-2025}German Ministry of Economic Affairs and Climate Action grant \#01MJ22003B\\
GANs and Semantics for Resilient, Flexible Robotic Production (GANResilRob)\\
Co-PI (from 10/2024), Collaborator\\
\years{2021-2025}German Ministry of Education and Research grant \#02L19C255\\
Competency Center Artificial Intelligence for Working and Living in the Karlsruhe Region (KARL)\\
Co-PI (from 10/2024), Collaborator\\
\years{2021-2024}German Ministry of Education and Research grant \#13GW0471B\\
Cognitively Assisted Laparoscopy: A Learning Robotic Assistance System for Surgical Grasping and Holding Tasks (Koala-Grasp)\\
Collaborator\\
\years{2021-2023}Baden-Württemberg Ministry for the Economy, Labor and Tourism, InvestBW grant \#BW1\_0079\\
Hybrid AI for Flexible Robotic Surface Treatment (RoboGrind)\\
Collaborator\\
\years{2020-2023}German Ministry of Education and Research grant \#16SV8406\\
A Multifunctional Service Robot to Support Care Professionals in the Hospital (HoLLiECares)\\
Collaborator\\
\years{2021-2024}German Ministry of Education and Research grant \#01IS20008C\\
AI-based Robot Calibration (KIRK)\\
Collaborator\\
\years{2019-2022}German Ministry of Education and Research grant \#01DR19001B\\
Imitation Learning from Human Demonstrations in Virtual Reality for Physical Human-Robot Interaction in Assistance Tasks (ILIAS)\\
Collaborator\\
\years{2017-2018}German Ministry of Economic Affairs and Energy, ZIM grant \#ZF4337302LF7\\
Frameworks and Technologies for the Monitoring, Analysis and Online Adaption of Industrial Robotic Production Processes (MonRob)\\
Collaborator\\
\years{2017-2020}German Ministry of Education and Research grant \#16SV7836\\
Robust Perception Skills for Robotic Household Assistance in the Context of Elderly Care (RoPHa)\\
Collaborator\\
\years{2019-2022}German Ministry of Education and Research grant \#02L17C556\\
Proactive Diagnosis and Conception of Collaborative Robot Deployment in Small and Medium-Sized Enterprises (ProBot)\\
Collaborator\\

\section*{Invited Talks}

\years{2024}

``GPUs sind das neue Öl: Die KI-Revolution im deutschen Mittelstand erfordert ein Umdenken bei der Bereitstellung kritischer Rechenkapazitäten''\\
Tag der digitalen Technologien, Fachforum ``Generative KI als Schlüssel zur Innovation!? Erfahrungen aus der (Arbeits-)Praxis''\\
German Ministry for Economic Affairs and Climate Action, Berlin, 8.10.2024

\medskip

``Verständlich, Sicher, Beherrschbar: Hybride KI für industrielle Anwendungen''\\
Innovative Arbeitswelten im Mittelstand\\
Projektträger Karlsruhe (PTKA) at Karlsruhe Institute of Technology (KIT), Karlsruhe, 20.09.2024

\medskip

``Ein interaktiver KI-Assistent für die Refabrikation mit Robotern: Datensparsam und flexibel dank hybrider KI''\\
KI-Showroom-Insight: KI und Produktion\\
FZI Research Center for Information Technology, Karlsruhe, 27.06.2024

\medskip

``What's in Your Head? Neurosymbolic AI for Intelligent Robots that we Understand'' (Keynote)\\
e-fellows IT Day\\
ZEIT Verlagsgruppe, McKinsey \& Company Inc, Stuttgart, 26.04.2024

\medskip

\years{2023}
``Intelligente Programmierassistenten für die industrielle Robotik''\\
Revolution der Robotik: Lösungen aus Baden-Württemberg\\
German Chamber of Commerce and Industry (IHK), Karlsruhe, 13.07.2023

\section*{Conference Activity}

\subsection*{Papers Presented}

\years{2024}
``RoboGrind: Intuitive and Interactive Surface Treatment with Industrial Robots'', IEEE International Conference on Robotics and Automation (ICRA), May 13-17

\medskip

``Human-AI Interaction in Industrial Robotics: Design and Empirical Evaluation of a User Interface for Explainable AI-Based Robot Program Optimization'', 57th CIRP Conference on Manufacturing Systems (CMS), May 29-31

\medskip

``BANSAI: Towards Bridging the AI Adoption Gap in Industrial Robotics with Neurosymbolic Programming'', 57th CIRP Conference on Manufacturing Systems (CMS), May 29-31

\medskip

``Domain-Specific Fine-Tuning of Large Language Models for Interactive Robot Programming'', European Robotics Forum (ERF), March 13-15

\medskip

\years{2023}

``EfficientPPS: Part-aware Panoptic Segmentation of Transparent Objects for Robotic Manipulation'', 56th International Symposium on Robotics (ISR Europe), September 26-27

\medskip

``Knowledge-Driven Robot Program Synthesis from Human VR Demonstrations'', 20th International Conference on Principles of Knowledge Representation and Reasoning (KR), September 2-8

\medskip

\years{2022}

``LapSeg3D: Weakly Supervised Semantic Segmentation of Point Clouds Representing Laparoscopic Scenes'', IEEE/RSJ International Conference on Intelligent Robots and Systems (IROS), October 23-27

\medskip

``Heuristic-Free Optimization of Force-Controlled Robot Search Strategies in Stochastic Environments'', IEEE/RSJ International Conference on Intelligent Robots and Systems (IROS), October 23-27

\medskip

``Localization and Tracking of User-Defined Points on Deformable Objects for Robotic Manipulation'', IEEE ICRA Workshop on Representing and Manipulating Deformable Objects, May 19

\medskip

``Robot Program Parameter Inference via Differentiable Shadow Program Inversion'', IEEE International Conference on Robotics and Automation (ICRA), May 30 - Jun 5

\medskip

\years{2020}

``Modulare, datengetriebene Roboterprogrammierung für die Lösung komplexer Handhabungsaufgaben in Alltagsumgebungen'', AAL-Kongress, September 29 - October 1

\section*{Departmental Talks}

\years{2024}
``Neurosymbolic Robot Programming: Bridging the Gap between Safe and Capable Robot Intelligence''\\
2024 EASE Fall School\\
University of Bremen/University of Michigan, Bremen, 13.11.2024

\medskip

\years{2022}
``Knowledge Representation and Reasoning in Industrial Robotics''\\
2022 EASE Fall School\\
University of Bremen, 23.09.2022

%----------------------------------------------------------------------------------------
%	TEACHING
%----------------------------------------------------------------------------------------

\section*{Teaching}

\subsection*{Guest Lectures}

\years{2025}

``AI-Enabled Robot Programming: Intuitive, Safe Human-Robot Interaction via Learning and Planning''\\
Robotik / Robotics\\
Hochschule für Technik Stuttgart, 17.01.2025

\medskip

``AI-Enabled Robot Programming: Intuitive, Safe Human-Robot Interaction via Learning and Planning''\\
Künstliche Intelligenz in der Produktion / Artificial Intelligence in Industrial Production\\
wbk Institute for Production Science, Karlsruhe Institute for Technology, 31.01.2025

%------------------------------------------------

\section*{Research Experience}

\years{2020-2025}
Model-based first-order robot program optimization with neural digital twins\\
Led research endeavor spanning 6 publicly funded research projects to develop the foundations of learned, modular, differentiable forward models of robot and environment dynamics, and their use in model-based robot program optimization.

\years{2022-2025}
Interactive, natural-language robot program synthesis\\
Led research endeavor spanning 4 publicly funded research projects to develop an AI assistant capable of synthesizing robot programs for complex, long-horizon tasks through natural-language dialogue with human users.

% \section*{Service to the Profession}

\vfill % Whitespace before final footer

%----------------------------------------------------------------------------------------
%	FINAL FOOTER
%----------------------------------------------------------------------------------------

% Any final footer text such as a URL to the latest version of this CV, last updated date, compiled in XeTeX, etc
\begin{center}
	\scriptsize
	Last updated: \today
\end{center}

%----------------------------------------------------------------------------------------

\end{document}
